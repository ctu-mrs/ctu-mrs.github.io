\documentclass{article}
\usepackage{tikz}   %TikZ is required for this to work.  Make sure this exists before the next line
% \usepackage{tikz-3dplot} %requires 3dplot.sty to be in same directory, or in your LaTeX installation

\usetikzlibrary{calc}

\usepackage[active,tightpage]{preview}  %generates a tightly fitting border around the work
\PreviewEnvironment{tikzpicture}
\setlength\PreviewBorder{2mm}
\usetikzlibrary{decorations.markings}
\usetikzlibrary{decorations.shapes}
\usetikzlibrary{calc,intersections,through,backgrounds}

\usetikzlibrary{arrows}
\usetikzlibrary{arrows.meta}
\usepackage[T1]{fontenc}
\usepackage{amsmath}

\usepackage{bm}  % bold italized fonts (for vectors)
\renewcommand\vec{\bm}
% \tdplotsetmaincoords{43}{13}{0}

\begin{document}

%%{ 

\newcommand{\drawDrone}[3]% x, y, z, size
{
  \coordinate (c) at #1;
  \pgfmathsetmacro{\armL}{1.5*#2}

  \path (c) ++(45:\armL) coordinate (r1);
  \path (c) ++(45+90:\armL) coordinate (r2);
  \path (c) ++(45+180:\armL) coordinate (r3);
  \path (c) ++(45+270:\armL) coordinate (r4);

  \draw [fill=white, draw=none] (r1) rectangle (r3);
  \draw [#3] (r1) -- (r3);
  \draw [#3] (r2) -- (r4);
  \draw [fill=white, #3] (r1) circle (#2);
  \draw [fill=white, #3] (r2) circle (#2);
  \draw [fill=white, #3] (r3) circle (#2);
  \draw [fill=white, #3] (r4) circle (#2);
}

%%}

\pgfmathsetmacro{\s}{1}
\pgfmathsetmacro{\f}{1.5}

\pgfmathsetmacro{\so}{3}
\pgfmathsetmacro{\d}{10}
\pgfmathsetmacro{\do}{(\d-\f)}

\pgfmathsetmacro{\ras}{0.2}
\pgfmathsetmacro{\ang}{atan(\so/(\d-\f))}
\pgfmathsetmacro{\ans}{0.65}

%start tikz picture, and use the tdplot_main_coords style to implement the display 
%coordinate transformation provided by 3dplot
\begin{tikzpicture}[scale=1,
    grid/.style={line width=0, gray!20}
  ]

  %%{ draw figure contents
  
  \begin{scope}
    \coordinate (O) at (0, 0);

    \coordinate (sA) at (-\f, \s);
    \coordinate (sB) at (-\f, -\s);

    \coordinate (soA) at (\do, \so);
    \coordinate (soB) at (\do, -\so);

    \coordinate (spA) at (intersection of soA--O and sA--sB);
    \coordinate (spB) at (intersection of soB--O and sA--sB);

    \draw[dashed] (-2*\f, 0) -- (\do+\f, 0);
    \draw[dashed] (0, 1.2*\so) -- (0, -1.2*\so);

    % sensor
    \draw (-\f, 0) rectangle ++(\ras, \ras); % right angle mark
    \draw[line width=3, blue] (sB) -- (sA) node [anchor=west] {sensor};
    \draw[<->] ($ (sB) + (-0.15, 0) $) -- ($ (sA) + (-0.15, 0) $) node [anchor=east] {$s$};
    % focal distance
    \draw[<->] (sB) ++(0, -0.15) -- node [midway, anchor=north] {$f$} ++(\f, 0);

    % object
    \draw (\d-\f, 0) rectangle ++(-\ras, \ras); % right angle mark
    \draw[line width=3, red] (soB) -- (soA) node [anchor=south east] {object};
    \draw[<->] ($ (soB) + (0.15, 0) $) -- ($ (soA) + (0.15, 0) $) node [anchor=west] {$s_o$};
    % object distance
    \draw[<->] (sB) ++(0, -0.75) -- node [midway, anchor=north] {$d$} ++(\d, 0);

    % object projection
    \draw[thin] (soA) -- (spA);
    \draw[thin] (soB) -- (spB);
    \draw[line width=1, orange] (spB) -- (spA) node [anchor=west, yshift=-2] {object projection};
    \draw[<->] ($ (spA) + (-0.15, 0) $) -- node [midway, anchor=south east] {$s_o'$} ($ (spB) + (-0.15, 0) $);

    % angles
    \draw (\ans, 0) arc (0 : \ang : \ans) node [midway, anchor=west, yshift=1] {$\alpha$};
    \draw (-\ans, 0) arc (180 : 180+\ang : \ans) node [midway, anchor=east, yshift=-1] {$\alpha$};

    % focal point
    \draw[<-, green!60!black] (O) -- ++(0.75, 3) node [anchor=west] {focal point};

  \end{scope}
  
  %%}

  \node at (-\f, 2*\f) {\footnotesize $\begin{aligned}
    \tan(\alpha) &= \frac{s_o}{2(d - f)} \\
    r_o &= \frac{r}{s}s_o'
  \end{aligned}$};

\end{tikzpicture}

\end{document}



