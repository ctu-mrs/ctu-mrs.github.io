\documentclass{article}
\usepackage{tikz}   %TikZ is required for this to work.  Make sure this exists before the next line
% \usepackage{tikz-3dplot} %requires 3dplot.sty to be in same directory, or in your LaTeX installation

\usetikzlibrary{calc}

\usepackage[active,tightpage]{preview}  %generates a tightly fitting border around the work
\PreviewEnvironment{tikzpicture}
\setlength\PreviewBorder{2mm}
\usetikzlibrary{decorations.markings}
\usetikzlibrary{decorations.shapes}

\usetikzlibrary{arrows}
\usetikzlibrary{arrows.meta}
\usepackage[T1]{fontenc}
\usepackage{amsmath}

\usepackage{bm}  % bold italized fonts (for vectors)
\renewcommand\vec{\bm}
% \tdplotsetmaincoords{43}{13}{0}

\begin{document}

%%{ 

\newcommand{\drawDrone}[3]% x, y, z, size
{
  \coordinate (c) at #1;
  \pgfmathsetmacro{\armL}{1.5*#2}

  \path (c) ++(45:\armL) coordinate (r1);
  \path (c) ++(45+90:\armL) coordinate (r2);
  \path (c) ++(45+180:\armL) coordinate (r3);
  \path (c) ++(45+270:\armL) coordinate (r4);

  \draw [fill=white, draw=none] (r1) rectangle (r3);
  \draw [#3] (r1) -- (r3);
  \draw [#3] (r2) -- (r4);
  \draw [fill=white, #3] (r1) circle (#2);
  \draw [fill=white, #3] (r2) circle (#2);
  \draw [fill=white, #3] (r3) circle (#2);
  \draw [fill=white, #3] (r4) circle (#2);
}

%%}

\pgfmathsetmacro{\gridMin}{0}
\pgfmathsetmacro{\gridMaxX}{4.0}
\pgfmathsetmacro{\gridMaxY}{2.0}

%start tikz picture, and use the tdplot_main_coords style to implement the display 
%coordinate transformation provided by 3dplot
\begin{tikzpicture}[scale=3, tdplot_main_coords,
    grid/.style={line width=0, gray!60}
  ]

  \coordinate (dm) at (1.3, 0.4);
  \coordinate (txtOff) at (0, 0.3);

  %%{ draw first figure contents
  
  \begin{scope}[xshift = 0]
    % draw the title
    \node at (\gridMaxX/2, \gridMaxY + 0.3) {situation at time $t_1$};

    % draw the grid
    \foreach \x in {\gridMin,0.25,...,\gridMaxX}
    {
        \draw[grid] (\x,\gridMin) -- (\x,\gridMaxY);
    }
    \foreach \y in {\gridMin,0.25,...,\gridMaxY}
    {
        \draw[grid] (\gridMin,\y) -- (\gridMaxX,\y);
    }
    
    %draw the main coordinate system axes
    \draw[thick, ->] (0,0,0) -- (0.25,0,0) node[anchor=north]{$\vec{e}_x$};
    \draw[thick, ->] (0,0,0) -- (0,0.25,0) node[anchor=east]{$\vec{e}_y$};
    % \draw[thick, ->] (0,0,0) -- (0,0,0.25) node[anchor=west]{$\vec{e}_z$};
    \draw[mark=*, mark size=1.0, black, mark options={fill=white}] plot coordinates { (0, 0, 0) };
    \draw[mark=*, mark size=0.2, black] plot coordinates { (0, 0, 0) }  node[anchor=east]{$\vec{e}_z$};

    \coordinate (mo1) at (2.3, 1);
    \coordinate (mc1) at ($ (mo1) + (dm) $);
    
    \drawDrone{(mo1)}{0.1}{thin}
    \drawDrone{(mc1)}{0.1}{thick, dotted, draw=green!60!black}
    \draw [red, very thick, ->] (mc1) -- (mo1) node [midway, red, yshift=-0.3cm, sloped] {disturbance};

    \node [align=center] at ($ (mo1) + (0, 0.5) $) {odometry\\\textbf{=}\\new command};
    \node [green!60!black, align=center] at ($ (mc1) + (0, 0.4) $) {prev.\\command};
  \end{scope}
  
  %%}

  %%{ draw second figure contents
  
  \begin{scope}[xshift = 5cm]
    % draw the title
    \node at (\gridMaxX/2, \gridMaxY + 0.3) {situation at time $t_2$};

    % draw the grid
    \foreach \x in {\gridMin,0.25,...,\gridMaxX}
    {
        \draw[grid] (\x,\gridMin) -- (\x,\gridMaxY);
    }
    \foreach \y in {\gridMin,0.25,...,\gridMaxY}
    {
        \draw[grid] (\gridMin,\y) -- (\gridMaxX,\y);
    }
    
    %draw the main coordinate system axes
    \draw[thick, ->] (0,0,0) -- (0.25,0,0) node[anchor=north]{$\vec{e}_x$};
    \draw[thick, ->] (0,0,0) -- (0,0.25,0) node[anchor=east]{$\vec{e}_y$};
    \draw[mark=*, mark size=1.0, black, mark options={fill=white}] plot coordinates { (0, 0, 0) };
    \draw[mark=*, mark size=0.2, black] plot coordinates { (0, 0, 0) }  node[anchor=east]{$\vec{e}_z$};

    \coordinate (mo2) at (1.0, 0.6);
    \coordinate (mc2) at (2.3, 1);
    
    \drawDrone{(mo2)}{0.1}{thin}
    \drawDrone{(mc2)}{0.1}{thick, dotted, draw=green!60!black}
    \draw [red, very thick, ->] (mc2) -- (mo2) node [midway, red, yshift=-0.3cm, sloped] {disturbance};

    \drawDrone{(3.6, 1.4)}{0.1}{thick, dashed, draw=blue}

    \node [align=center] at ($ (mo2) + (0, 0.5) $) {odometry\\\textbf{=}\\new command};
    \node [green!60!black, align=center] at ($ (mc2) + (0, 0.4) $) {prev.\\command};
    \node [blue, align=center] at (3.6, 1.8) {intended\\position};
  \end{scope}
  
  %%}

\end{tikzpicture}

\end{document}



